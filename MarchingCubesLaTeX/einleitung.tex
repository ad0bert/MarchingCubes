\chapter{Einleitung}

\section{Aufgabenstellung}
In der medizinischen Diagnostik wird im Gegensatz zu CAD-Konstruktionen die dreidimensionale
Gestalt anatomischer Details aus Volumsbildern abgeleitet. Durch vorangegangene Segmentierung werden binäre Objekte erzeugt, d.h. das Objekt ist wie eine Lego-Figur aufgebaut. Mit dem Marching Cubes Algorithmus wird aus diesem binären Volumen eine Oberfläche, welche aus Dreiecken besteht, aufgebaut. Diese Oberfläche wird in einem binären STL-Format persistiert und anschließend mithilfe von generischem Rendering als 3D-Objekt dargestellt.\\\\
Anforderungen: C/C++ Implementierung des MC-Algorithmus (Matlab-Version vorhanden), Konversion in STL-Format, OpenGl Visualisierung.

\section{Motivation}
Da moderne Grafikchips auf die Darstellung von polygonalen Modellen ausgelegt sind, ist es sinnvoll, die aus der medizinischen Diagnostik erhaltenen Voxel-Modelle für spätere Verarbeitung in diese Form zu überführen. Ein weiterer Vorteil neben der vereinfachten Verarbeitung und Darstellung von Polygonen liegt in dem vergleichsweise geringen Speicherbedarf eines solchen Objektes.

\section{Zielsetzung}
Ziel dieser Arbeit ist es, die aus bildgebenden Verfahren der Medizin erhaltenden Voxel-Mengen mithilfe des Marching Cubes Algorithmus in eine polygonale Darstellung zu überführen. Als Input werden die Daten, welche von dem Programm Analyze 7.5\footnote{https://rportal.mayo.edu/bir/} erzeugt werden verwendet. Im Speziellen handelt es sich hierbei um die Formate Image (.img) und Header (.hdr). Die Voxel-Menge, welche in der Image-Datei abgelegt ist, wird ausgelesen und mithilfe des Marching Cubes Algorithmus in Polygone zerlegt. Nach erfolgreicher Umwandlung wird die erhaltene Datenmenge via OpenGL dargestellt. Des Weiteren soll das Modell als STL-Datei exportiert werden können. Die gesamte Umsetzung erfolgt in der Programmiersprache C++. 
