
\chapter{Kurzfassung}

In der computergestützten Bildverarbeitung gibt es diverse Möglichkeiten für die Darstellung von dreidimensionalen Objekten. Die wohl am weitesten verbreitete Darstellungsform ist die polygonale Darstellung. Diese Form der Aufbereitung zerlegt ein gegebenes Objekte in Dreiecke. 
\\\\
Ein weiteres Verfahren ist die Methode der Modellierung via einer so genanten Voxel-Datenmenge. Jedoch birgt diese, im Bezug auf die digitale Verarbeitung, einige Nachteile gegenüber der polygonalen Darstellungsform. Vordergründige Probleme hierbei sind der vergleichsweise hohe Speicherverbrauch der Modelle, die Visualisierung benötigt länger und Objektmanipulationen erweisen sich als schwieriger.
\\\\
Da in der Medizin im Bereich der bildgebenden Systeme wie der Computertomografie von Natur aus solche Voxel-Modelle erzeugt werden, besteht die Anforderung auch diese nach Möglichkeit schnell und Aussagekräftig darzustellen.
\\\\
Um diesen Anforderungen an die Darstellung gerecht zu werden bietet sich der so genante Marching-Cubes Algorithmus an. Dieser ermöglicht es eine Voxel-Datenmenge in eine polygonale Darstellung zu überführen. 
