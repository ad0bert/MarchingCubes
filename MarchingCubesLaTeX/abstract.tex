\chapter{Abstract}

Considering the computer based image processing there are multiple possibilities to represent three-dimensional objects. The most common way to illustrate these objects is the polygonal approach. This approach fragments an object into triangles.
\\\\
An other possible procedure to model objects is to represent them as a voxel grid. But if we consider the ability to process this kind of representation we have to face some disadvantages. The main problems are: the model needs a comparatively high amount of disk space, it takes much longer to show the image and it is difficult to perform manipulations on the object.
\\\\
In the field of Medical imaging such as computed tomography creates such voxel models, these should be presented quickly and meaningfully.
\\\\
To implement these requirements on the presentation we have to transform voxel grids into polygon objects. The so called marching cubes algorithm can achieve this goal.
\\\\
This thesis gives a quick overview of the Marching Cubes Algorithm and tries to explain it in an easy way. To show the functionality it provides an C/C++ implementation of the Algorithm. It also provides the possibility to represent a generated model via OpgenGL and shows how to export a polygonal object as .stl File.