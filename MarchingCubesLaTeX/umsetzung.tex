\chapter{Umsetzung}

\section{Marching Cubes}

\subsection{Ansatz}

\subsection{Implementierung}

\section{File Formate}

\subsection{STereoLithography (.stl)}
\ref{prog:generateSTL}


\begin{program}
	\caption{Generierung einer STL-Datei}
	\label{prog:generateSTL}
	\begin{CCode}
	bool marchingCubes::GenerateStlFile(std::string path){
		FILE *fptr = NULL;
		fprintf(stderr, "Writing triangles ...\n");
		if ((fptr = fopen(path.c_str(), "a+b")) == NULL) {
			fprintf(stderr, "Failed to open output file\n");
			return false;
		}
		char fileHeader[81] = "solid Test Head";
		char bytes[3] = { 0x00, 0x00 };
		fwrite(&fileHeader, sizeof(fileHeader)-1, 1, fptr);
		fwrite(&ntri, sizeof(int), 1, fptr);
		for (int i = 0; i < ntri; i++) {
			fwrite(&tri[i].n[0], sizeof(float), 3, fptr);
			for (int k = 0; k < 3; k++)  {
				fwrite(&tri[i].p[k], sizeof(float), 3, fptr);
			}
			fwrite(bytes, 2, 1, fptr);
		}
		fclose(fptr);
		return true;
	}
	\end{CCode}
\end{program}

\section{OpenGL}

\begin{CCode}
	void marchingCubes::CalcNormal(TRIANGLE &tri){
		XYZ U;
		XYZ V;
		U.x = tri.p[1].x - tri.p[0].x;
		U.y = tri.p[1].y - tri.p[0].y;
		U.z = tri.p[1].z - tri.p[0].z;
		
		V.x = tri.p[2].x - tri.p[0].x;
		V.y = tri.p[2].y - tri.p[0].y;
		V.z = tri.p[2].z - tri.p[0].z;
		
		tri.n[0].x = (U.y * V.z) - (U.z * V.y);
		tri.n[0].y = (U.z * V.x) - (U.x * V.z);
		tri.n[0].z = (U.x * V.y) - (U.y * V.x);
	}
\end{CCode}