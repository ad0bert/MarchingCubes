
\chapter{Kurzfassung}

In der computergestützten Bildverarbeitung gibt es diverse Möglichkeiten für die Darstellung von dreidimensionalen Objekten. Die wohl am weitesten verbreitete Darstellungsform ist die polygonale Darstellung. Diese Form der Aufbereitung zerlegt Objekte in Dreiecke. Ein weiteres Verfahren ist die Methode der Modellierung via Voxel-Datenmenge. Jedoch birgt diese im Bezug auf die digitale Verarbeitung einige Nachteile gegenüber der polygonalen Darstellungsform. Vordergründige Probleme hierbei sind der vergleichsweise hohe Speicherverbrauch der Modelle, die Visualisierung benötigt länger und Objektmanipulationen erweisen sich als schwierig. Da allerdings in der Medizin im Bereich der bildgebenden Systeme wie die Computertomografie von Natur aus solche Modelle erzeugt werden müssen auch diese nach Möglichkeit schnell und Aussagekräftig dargestellt werden.

Um nun diese Anforderung an die Darstellung umzusetzen bietet sich der so genante Marching-Cubes Algorithmus an welcher es ermöglicht eine Voxel-Datenmenge in eine polygonale Darstellung zu überführen. 
